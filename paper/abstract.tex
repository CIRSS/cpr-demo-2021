A growing number of journal publishers verify computational artifacts as part of the peer-review process. In support of these efforts Whole Tale aims to enable verifiers to observe and evaluate how automated computational workflows produce intermediate and final artifacts. The Whole Tale Comprehensible Provenance Record (CPR) Toolkit is a suite of tools for recording, storing, querying, and visualizing the provenance of artifacts produced by a run of a computational workflow. A key objective of the toolkit is to make provenance easily comprehensible, not to systems programmers, but to practitioners of a research domain seeking to understand how the computational artifacts associated with a study in that domain were obtained. CPR makes traces of computational workflows transparent and actionable, and enables verifiers and others to judge whether the computations were performed appropriately.