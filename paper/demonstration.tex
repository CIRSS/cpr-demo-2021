\section{Demonstration}

The CPR demo is provided as a Git repository and associated Docker image that enable the examples to be run on any Linux, macOS, or Windows-based system with Git, Docker, and GNU Make installed.  Each example uses the CPR toolkit to record OS-level provenance information from a run of different computational workflow, to load a Blazegraph instance with the resulting CPR traces, and to produce a set of reports and visualizations via SPARQL queries.

A Makefile in the top directory of the demo repository provides targets for pulling the Docker image from Dockerhub (\emph{pull-image}), building the Docker image locally (\emph{build-image}), for running the examples (\emph{run-examples}), and for deleting all of the reports, visualizations, and other artifacts generated for each example (\emph{clean-examples}). Because the expected results are included in the repository, successful reproduction of the example products is demonstrated by issuing the commands \emph{make clean-examples} and \emph{make run-examples} and confirming that \emph{git diff} reports no differences.
 
\subsection{Queries and Visualizations}

Queries and visualizations produced for each example included the following. A question phrased in English summarizes the intent of each.

\emph{What files are employed as inputs and produced as outputs of the run?}

\emph{What are the input and output data files for each process in the run?}

\emph{Which files input to a run are used to produce a particular output file?}

\emph{Which output artifacts are affected by a particular input file?}

\emph{What programs and script invocations occur as part of the run?}

\emph{What programs and script invocations contribute to the production of a particular output artifact?}

\emph{What constraints on the order of execution of different processes does the observed flow of data imply?}

\subsection{Observations}

A key challenge in making provenance useful to researchers and verifiers alike is highlighting the small subset of recorded events of direct relevance to the scientific purpose of an execution.

An execution of a one-line Python 3 script that prints "Hello World" can involve reading as many as X different files from disk in addition to the users' single-line Python file.
 
CPR minimizes such provenance "noise" using a CPR profile that assigns distinct roles to files loaded from particular locations on the system.
