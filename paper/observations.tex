\section{Observations}

The computations and queries demonstrated here highlight a key challenge in making provenance useful to domain researchers and verifiers: revealing the small subset of recorded events that are of direct relevance to the scientific purpose of an overall computational workflow. At a low level, execution of even a one-line Python 3 script that prints ``Hello World'' can involve reading tens of different files from disk in addition to the single-line Python file that the user supplied. CPR minimizes such provenance ``noise'' using SPARQL queries that select files and processes with particular relationships to other files and processes, optionally informed further by a user-provided run profile that assigns distinct roles to files loaded from particular locations on the system.  For example, it can be useful to hide processes that do not themselves read or write data files; a bash script that serves only to invoke other programs that do process data files can be masked even in the absence of a profile. The bash script listed in \figref{fig-script} is not depicted graphically in \figref{subfig-blackbox} and \figref{subfig-whitebox} because these queries filter out processes that do not perform I/O on data files.

A second key challenge to making provenance useful to domain specialists is providing vocabularies that convey the significance of particular processes and data artifacts in domain-specific terms. PROV and ProvONE provide essential abstract base classes from which more meaningful provenance vocabularies can be derived.  Domain researchers--and the verifiers of computations reported in their papers--likely will find views of provenance employing such specialized vocabularies the most useful. Nevertheless, the base classes are essential for performing general queries that must succeed on traces captured from any domain, e.g. to answer the question, \emph{What are all the files--data files, scripts, executables, shared libraries, etc--that must be archived and restored later to repeat the computation?} By describing computations in terms of files used to store data and processes executed on real computers, the CPR vocabulary provides a set of concepts intermediate to the more general ones comprising PROV and ProvONE, and the more specific concepts of domain-specific vocabularies. 

Moreover, deriving the CPR vocabulary from existing standard vocabularies provides multiple options when depositing data and its provenance in public repositories such as DataONE. Because Blazegraph eagerly infers triples implied by RDF schema declarations, exporting provenance as simply as PROV, or as ProvONE, or as a combination of PROV, ProvONE, and the CPR vocabularies, is as simple as performing a trivial CONSTRUCT query that extracts triples that already exist in the RDF dataset. Finally, much as common base classes in object-oriented programming languages make it convenient to work with collections of objects that are instances of more specialized classes, we expect that access to the PROV, ProvONE, and CPR vocabularies when querying provenance expressed in more specialized vocabularies will in many cases simplify those queries and make them more transparent and reusable.



